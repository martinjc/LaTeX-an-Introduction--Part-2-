\documentclass[12pt,twoside,a4paper]{article}
\usepackage{lipsum}
\usepackage{setspace}
\usepackage{multirow}
\usepackage{tabularx}
\usepackage{graphicx}
\usepackage{amsmath}
\usepackage{subcaption}

\newcommand{\testcmd}[3]{This is a test command with arguments: #1, #2, #3}


\setlength{\parindent}{0pt}

\title{A Test \LaTeX\ Document}
\author{Martin Chorley}
\date{February 2013}

\begin{document}

\maketitle

\begin{abstract}
\lipsum[2]
\end{abstract}

\section*{Introduction}

\testcmd{Hello}{World}

\onehalfspacing
{\Huge Hello World!} This is       my       first       \LaTeX\     document.
This is not a new paragraph.

\textsc{This is a new paragraph.}

"Here is something in quotation marks"

``Here is something with grave accents and apostrophes''

\begin{center}{\tiny Here is some centred text}\end{center}

\begin{itemize}
	\item A list item
	\item Another item
	\item Some more item
	\item The end of the list
\end{itemize}

\begin{enumerate}
	\item First item in our list
		\begin{enumerate}
			\item First item in the sub-list
			\item Another item
					\begin{itemize}
						\item First item in the sub-sub-list
						\item Another item
							\begin{enumerate}
								\item First item in the sub-sub-list
								\item Another item
							\end{enumerate}
					\end{itemize}
		\end{enumerate}
	\item Yet more list items
	\item And one for luck
\end{enumerate}


\lipsum[1-4]

\section{Related Work}
\singlespacing

\lipsum[5-6]

\begin{table}[ht]

	\begin{tabular}{ | r | l | }
    	\hline
    	\multicolumn{2}{|c|}{Table Heading} \\
    	\hline
    		some text & other text \\
			this text & more text \\
    	\hline
	\end{tabular}

\end{table}

	\begin{figure*}[htb]
	    \centering
	    \begin{subfigure}[b]{0.3\textwidth}
	        \includegraphics[width=\textwidth]{img/lights}
	        \caption{Some lights}
	        \label{fig:lights}
	    \end{subfigure}
	    \begin{subfigure}[b]{0.3\textwidth}
	        \includegraphics[width=\textwidth]{img/bench}
	        \caption{A bench}
	        \label{fig:bench}
	    \end{subfigure}
	    \caption{Some lights and a bench}
	    \label{fig:subfigexample}
	\end{figure*}
	
Figure~\ref{fig:subfigexample} on Page~\pageref{fig:subfigexample} shows us two things. In Figure~\ref{fig:lights} we see some lights, and in Figure~\ref{fig:bench} we see a bench.

\lipsum[5-6]

\section{Tables}

Some tables

\vspace{0.2in}

\begin{tabular}{| r | c |}
\hline
first cell & second column \\
\hline
second row & last cell \\
\hline
\end{tabular}

\vspace{0.2in}




\vspace{0.2in}

\begin{tabular}{ | r | c | l | }
    \hline
    \multicolumn{3}{|c|}{Table Heading} \\
\hline
    \multirow{2}{*}{this text} & some text & other text \\
     & more text & other more text \\
    \hline
\end{tabular}

\vspace{0.2in}

\begin{tabularx}{\textwidth}{ | X | X | X | }
    \hline
    \multicolumn{3}{|c|}{Table Heading} \\
\hline
    \multirow{2}{*}{this text} & some text & other text \\
     & more text & other more text \\
    \hline
\end{tabularx}

\section{Images}

\setlength\fboxrule{6pt}
\setlength\fboxsep{0pt}
\fbox{\includegraphics[width=0.5\textwidth]{img/django.jpg}}



\section{Maths}

\[ \int_0^\infty x^2\,dx \]

\[ \int_0^\infty x^2\,\mathrm{d}x\]

\[  \left(\frac{x^2}{y^3}\right) \]

\[ \Bigg[ 0\Bigg] \]

\[ [ 0 ] \]

\[ 50 \text{ apples} \]

\begin{equation}
\label{eq:myequation}
    \lim_{x \to \infty} \exp{-x} = 0
\end{equation}

We can refer to our equation~\ref{eq:myequation}, or even use~\eqref{eq:myequation} to get a different style of reference.

\end{document}